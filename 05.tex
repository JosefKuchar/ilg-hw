\begin{figure}
\section*{Příklad 5}
Pomocí Gram-Schmimdtova ortogonalizačního procesu najděte ortogonální a pak i ortonormální bázi prostoru generovaného vektory \\\\
Výpočet ortogonální báze
$$
  \vec{a_1} = [1, 2, -3, 1],
  \vec{a_2} = [2,3, -2, 1],
  \vec{a_3} = [1, -1, 9, -2]
$$
Výpočet $\vec{b_1}$
$$
  \vec{b_1} = \vec{a_1} = [1, 2, -3, 1]
$$
Výpočet $\vec{b_2}$
\begin{equation*}
    \begin{aligned}
  \vec{b_2} &= \vec{a_2} + r \vec{b_1} \\
  \vec{b_1} \cdot \vec{b_2} &= \vec{b_1} \cdot \vec{a_2} + r \vec{b_1} \cdot \vec{b_1} \\
  r &= - \frac{\vec{b_1} \cdot \vec{a_2}}{\vec{b_1} \cdot \vec{b_1}}
  = - \frac{[1, 2, -3, 1] \cdot [2,3, -2, 1]}{[1, 2, -3, 1] \cdot [1, 2, -3, 1]}
  = - \frac{2 + 6 + 6 + 1}{1 + 4 + 9 + 1}
  = - \frac{15}{15}
  = -1 \\
  \vec{b_2} &= \vec{a_2} + r \vec{b_1} = [2,3, -2, 1] - 1 [1, 2, -3, 1] = [1, 1, 1, 0] \\
  \vec{b_1} \cdot \vec{b_2} &= [1, 2, -3, 1] \cdot [1, 1, 1, 0] = 0
    \end{aligned}
\end{equation*}
Výpočet $\vec{b_3}$
\begin{equation*}
    \begin{aligned}
  \vec{b_3} &= \vec{a_3} + s \vec{b_1} + u \vec{b_2} \\
  \vec{b_1} \cdot \vec{b_3} &= \vec{b_1} \cdot \vec{a_3} + s \vec{b_1} \cdot \vec{b_1} + u \vec{b_1} \cdot \vec{b_2} \\
  s &= - \frac{\vec{b_1} \cdot \vec{a_3}}{\vec{b_1} \cdot \vec{b_1}}
  = - \frac{[1, 2, -3, 1] \cdot [1, -1, 9, -2]}{[1, 2, -3, 1] \cdot [1, 2, -3, 1]} 
  = - \frac{1 -2 -27 -2}{1 + 4 + 9 + 1} = -\frac{-30}{15} = 2 \\
  \vec{b_2} \cdot \vec{b_3} &= \vec{b_2} \cdot \vec{a_3} + s \vec{b_2} \cdot \vec{b_1} + u \vec{b_2} \cdot \vec{b_2} \\
  u &= - \frac{\vec{b_2} \cdot \vec{a_3}}{\vec{b_2} \cdot \vec{b_2}}
  = -\frac{[1, 1, 1, 0] \cdot [1, -1, 9, -2]}{[1, 1, 1, 0] \cdot [1, 1, 1, 0]}
  = -\frac{1 -1 + 9 + 0}{1 + 1 + 1 + 0} = -\frac{9}{3} = -3 \\
  \vec{b_3} &= \vec{a_3} + s \vec{b_1} + u \vec{b_2}
  = [1, -1, 9, -2] + 2 \cdot [1, 2, -3, 1] -3 \cdot [1, 1, 1, 0]
  = [0, 0, 0, 0]
    \end{aligned}
\end{equation*}
Poslední vektor vyšel nulový, což znamená, že vektory  $\vec{a_1},\vec{a_2}, \vec{a_3}$  jsou lineárně závislé \\\\
Ortogonální báze
$$ \vec{b_1} = [1, 2, -3, 1], \vec{b_2} = [1, 1, 1, 0]$$
Výpočet ortonormální báze
\begin{equation*}
    \begin{aligned}
        \vec{c_1} &= \frac{\vec{b_1}}{||\vec{b_1}||}
        = \frac{[1, 2, -3, 1]}{\sqrt{1^2 + 2^2 + (-3)^2 + 1^2}}
        = \left[\frac{1}{\sqrt{15}}, \frac{2}{\sqrt{15}}, \frac{-3}{\sqrt{15}}, \frac{1}{\sqrt{15}}\right]
        = \left[\frac{\sqrt{15}}{15}, \frac{2\sqrt{15}}{15}, \frac{-\sqrt{15}}{5}, \frac{\sqrt{15}}{15}\right] \\
        \vec{c_2} &= \frac{\vec{b_2}}{||\vec{b_2}||}
        = \frac{[1, 1, 1, 0]}{\sqrt{1^2 + 1^2 + 1^2 + 0^2}}
        = \left[\frac{1}{\sqrt{3}}, \frac{1}{\sqrt{3}}, \frac{1}{\sqrt{3}},0\right]
        = \left[\frac{\sqrt{3}}{3}, \frac{\sqrt{3}}{3}, \frac{\sqrt{3}}{3},0\right]
    \end{aligned}
\end{equation*}
Ortonormální báze
$$ \vec{c_1} = \left[\frac{\sqrt{15}}{15}, \frac{2\sqrt{15}}{15}, \frac{-\sqrt{15}}{5}, \frac{\sqrt{15}}{15}\right], \vec{c_2} = \left[\frac{\sqrt{3}}{3}, \frac{\sqrt{3}}{3}, \frac{\sqrt{3}}{3},0\right]$$
\end{figure}