\newpage
\section*{Příklad 4}
Najděte $LU$ rozklad matice
$$
A =
\begin{pmatrix}
  1  & -1 & 3   \\
  2  & 0  & 1   \\
  -1 & 5  & -12
\end{pmatrix}
$$
Pak pomocí nalezeného $LU$ rozkladu najděte řešení soustavy rovnic

\begin{equation*}
  \sysdelim..\systeme{
    x - y + 3z = 5,
    2x + z = 1,
    -x + 5y -12z = -22
  }
\end{equation*}
K nalezení $LU$ rozkladu použijeme metodu úpravy matice na trojúhelníkový tvar, přidáme si druhou matici, ze které vznikne L matice.
$$
\begin{pmatrix}
  1  & -1 & 3   \\
  2  & 0  & 1   \\
  -1 & 5  & -12
\end{pmatrix}
\qquad
\begin{pmatrix}
  1 & 0 & 0 \\
  . & 1 & 0 \\
  . & . & 1 
\end{pmatrix}
$$
K "vynulování" $a_{21}$ přičteme vynásobíme první řádek čísem $(-2)$ a přičteme k druhému. Do naší vedlejší matice si na pozici $l_{21}$ zapíšeme číslo opačné tedy $2$
$$
\begin{pmatrix}
  1  & -1 & 3   \\
  0  & 2  & -5   \\
  -1 & 5  & -12
\end{pmatrix}
\qquad
\begin{pmatrix}
  1 & 0 & 0 \\
  2 & 1 & 0 \\
  . & . & 1 
\end{pmatrix}
$$
Nyní pro prvek $a_{31}$, k $3$ řádku přičteme $1\times$ první
$$
\begin{pmatrix}
  1  & -1 & 3   \\
  0  & 2 & -5   \\
  0 & 4  & -9
\end{pmatrix}
\qquad
\begin{pmatrix}
  1 & 0 & 0 \\
  2 & 1 & 0 \\
  -1 & . & 1 
\end{pmatrix}
$$
Poslední zbývá prvek $a_{32}$, k $3$ řádku přičteme $(-2)\times$ druhý
$$
\begin{pmatrix}
  1  & -1 & 3   \\
  0  & 2 & -5   \\
  0 & 0  & 1
\end{pmatrix}
\qquad
\begin{pmatrix}
  1 & 0 & 0 \\
  2 & 1 & 0 \\
  -1 & 2 & 1 
\end{pmatrix}
$$
Nyní máme obě matice,
$$
U =
\begin{pmatrix}
  1  & -1 & 3   \\
  0  & 2 & -5   \\
  0 & 0  & 1
\end{pmatrix}
\qquad
L =
\begin{pmatrix}
  1 & 0 & 0 \\
  2 & 1 & 0 \\
  -1 & 2 & 1 
\end{pmatrix}
$$
K řešení soustavy využijeme asociativitu násobení matic.
$$
\begin{pmatrix}
  1 & 0 & 0 \\
  2 & 1 & 0 \\
  -1 & 2 & 1 
\end{pmatrix}
\cdot
\begin{pmatrix}
  1  & -1 & 3   \\
  0  & 2 & -5   \\
  0 & 0  & 1
\end{pmatrix}
\cdot
\begin{pmatrix}
  x \\
  y \\
  z \\
\end{pmatrix}
=
\begin{pmatrix}
  5 \\
  1 \\
  -22 \\
\end{pmatrix}
$$
Máme soustavu ve tvaru $L \cdot U \cdot \bar x = \bar b$ \\
Nejprve položíme $U\cdot \bar x = \bar y$
$$
\begin{pmatrix}
  1  & -1 & 3   \\
  0  & 2 & -5   \\
  0 & 0  & 1
\end{pmatrix}
\cdot
\begin{pmatrix}
  x \\
  y \\
  z \\
\end{pmatrix}
=
\begin{pmatrix}
  x_1 \\
  y_1 \\
  z_1 \\
\end{pmatrix}
$$
Nyní vyřešíme $L\cdot \bar y = \bar b$
$$
\begin{pmatrix}
  1 & 0 & 0 \\
  2 & 1 & 0 \\
  -1 & 2 & 1 
\end{pmatrix}
\cdot
\begin{pmatrix}
  x_1 \\
  y_1 \\
  z_1 \\
\end{pmatrix}
=
\begin{pmatrix}
  5 \\
  1 \\
  -22 \\
\end{pmatrix}
$$
Přepíšeme do soustavy
\begin{equation*}
  \sysdelim..\systeme{
    x_1 = 5,
    2x_1 + y_1 = 1,
    -x_1 + 2y_1 - z_1 = -22
  }
\end{equation*}
$$2 \cdot 5 + y_1 = 1 \implies y_1 = -9$$
$$-(5) + 2 \cdot (-9) + z_1 = -22 \implies z_1 = 1$$
Nyní vyřešíme $U \cdot \bar x = \bar y$

$$
\begin{pmatrix}
  1  & -1 & 3   \\
  0  & 2 & -5   \\
  0 & 0  & 1
\end{pmatrix}
\cdot
\begin{pmatrix}
  x \\
  y \\
  z \\
\end{pmatrix}
=
\begin{pmatrix}
  5 \\
  -9 \\
  1 \\
\end{pmatrix}
$$
Přepíšeme do soustavy
\begin{equation*}
  \sysdelim..\systeme{
    x - y + 3z = 5,
      2y - 5z = -9,
    z = 1
  }
\end{equation*}
$$2y - 5 \cdot 1 = -9 \implies y = -2$$
$$x - 1 \cdot (-2) + 3 \cdot 1 = 5 \implies x = 0$$
Řešením soustavy je usporádaná trojice $\left[0, -2, 1\right]$.